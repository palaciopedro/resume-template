\documentclass[a4paper,10pt]{article}

\usepackage[utf8]{inputenc}
\usepackage[T1]{fontenc}
\usepackage[brazil]{babel}
\usepackage{geometry}
\usepackage{parskip}
\usepackage{hyperref}
\usepackage{titlesec}
% Para quebras de linha em URLs longas (não use \href{}, use \url{} para URLs longas)
% \usepackage{xurl}

% --- CONFIGURAÇÃO DO DOCUMENTO ---
% Define margens da página para maximizar espaço de conteúdo
\geometry{top=1.0cm, bottom=1.0cm, left=1.0cm, right=1.0cm}

% Remove números de página e cabeçalhos para visual limpo do currículo
\pagestyle{empty}

% Metadados do PDF - personalize com suas informações
\hypersetup{
    pdftitle={Pedro Palacio - Currículo},
    pdfauthor={Pedro Palacio},
    colorlinks=true,
    linkcolor=black,
    urlcolor=black,
    citecolor=black,
    bookmarksdepth=1 
}

% Desabilita numeração das sections
\setcounter{secnumdepth}{0}

% Formata os cabeçalhos de cada section e coloca uma linha em baixo
\titleformat{\section}
{\Large\bfseries}
{}
{0em}
{}
[\titlerule\vspace{0.5ex}]

% --- INÍCIO DO DOCUMENTO ---
\begin{document}

% --- CABEÇALHO ---
% Substitua com suas informações pessoais
\begin{center}
    {\LARGE \textbf{Pedro Palacio Bahniuk}} 
    \\ [0.1cm]
    Mandaguari, Paraná
    {\textbullet}
    Email: \href{mailto:pedropbhk@gmail.com}{pedropbhk@gmail.com} 
    {\textbullet}
    \href{https://www.linkedin.com/in/pedro-palacio}{linkedin.com/in/pedro-palacio} 
    {\textbullet}
    \href{https://github.com/palaciopedro}{github.com/palaciopedro}
\end{center}

% --- SEÇÕES ---

\section{Educação}
    % Formação mais recente primeiro
    \subsection*{\texorpdfstring{
            \textbf{Universidade Estadual de Maringá} \hfill Maringá, Paraná
        }{
            Nome da Sua Universidade (Educação) -- Localização
        }}
    \textit{Bacharelado em Engenharia de Software \hfill Mar 2026 - Mar 2030 }

\section{Sobre mim}\\
        Estudante de Engenharia de Software na Universidade Estadual de Maringá com sólida formação em Python. Experiência na criação de aplicações simples para uso pessoal e controle de versão com Git. Atualmente, aprimorando meus conhecimentos em Java e desenvolvimento back-end. Busco uma oportunidade de estágio para contribuir com minhas habilidades técnicas e adquirir experiência na área.

\section{Habilidades}
    \begin{itemize}
        \item \textbf{Programação:} Python, Java
        \item \textbf{Desenvolvimento Web:} HTML5, CSS3
        \item \textbf{Database:} PostgreSQL
        \item \textbf{Ferramentas:} Git, GitHub
        \item \textbf{Idiomas:} Português (Nativo), Inglês (Avançado)
    \end{itemize}

\section{Projects}
\textbf{Website Pessoal} \\
Desenvolvi um site responsivo usando HTML e CSS para apresentar informações profissionais e links de contato. Implementei um design de layout limpo e responsivo para dispositivos móveis. Hospedado no GitHub Pages.

\textbf{Task Tracker CLI – Python} \\
Desenvolvimento de aplicação em linha de comando (CLI) para gerenciamento de tarefas, permitindo operações de CRUD (criar, listar, atualizar e remover). Implementação de persistência de dados em arquivo JSON e manipulação de argumentos via terminal sem uso de bibliotecas externas. Código estruturado de forma modular visando organização e manutenibilidade.

\end{document}